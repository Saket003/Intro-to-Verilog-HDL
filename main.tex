\documentclass{article}
\usepackage{graphicx} % Required for inserting images
\usepackage{listings}
\usepackage{color}

\title{Verilog HDL}
\author{Saket Kandoi}
\date{March 2023}

\definecolor{dkgreen}{rgb}{0,0.6,0}
\definecolor{gray}{rgb}{0.5,0.5,0.5}
\definecolor{mauve}{rgb}{0.58,0,0.82}

\lstset{frame=tb,
  language=Verilog,
  aboveskip=3mm,
  belowskip=3mm,
  showstringspaces=false,
  columns=flexible,
  basicstyle={\small\ttfamily},
  numbers=none,
  numberstyle=\tiny\color{gray},
  keywordstyle=\color{blue},
  commentstyle=\color{dkgreen},
  stringstyle=\color{mauve},
  breaklines=true,
  breakatwhitespace=true,
  tabsize=3
}

%Reference - https://stackoverflow.com/questions/3175105/inserting-code-in-this-latex-document-with-indentation
%https://mirror.niser.ac.in/ctan/macros/latex/contrib/listings/listings-devel.pdf

\begin{document}

\maketitle

\section{Terminology}
\begin{itemize}
    \item HDL: Hardware Description Language 
    \item Behavior Modelling: Output in relation to inputs, only functionality
    \item Structural Modelling: Functionality and structure of the circuit
    \item RTL Synthesis: Translation then Optimization to Gate-level
\end{itemize}

\section{Syntax}
\begin{itemize}
    \item Case-sensitive, keywords are lower-case
    \item Whitespace insensitive

\item Module Structure
 \begin{lstlisting}
 \\This is a comment
 \*
 This too
 *\
module module_name(port_list);
    <port declarations>
    <data type declarations>
    <circuit functionality>
    <timing specifications>
endmodule
\end{lstlisting}

\item Port Declarations
\begin{lstlisting}
    input [7:0] ina,inb;
    input clk,aclr;
    output [15:0] out;
    //inout for bidirectional ports
    //Buses [n:0] have size: n+1-Bit
\end{lstlisting}
\begin{lstlisting}
    //Alternate Method
    module mult_acc
    (
    input [7:0] ina,inb,
    input clk,aclr,
    output [15:0] out
    );
    ...
    endmodule
\end{lstlisting}

\item Data Type Declarations\\
Two Types - 
    \begin{itemize}
        \item Net: Activity flows
        \item Variable: Element to store data temp
    \end{itemize}

\begin{center}
\begin{tabular}{ |c|c| } 
 \hline
 wire & Node or Connection  \\ 
 tri & Node three-state  \\ 
 supply0 and supply1 & Logic Values  \\ 
 \hline
\end{tabular}
\end{center}

\begin{lstlisting}
    //<data_type> [MSB:LSB] <signal name>;
    //<data_type> [LSB:MSB] <signal name>;
    wire [7:0] out;
    tri enable;
\end{lstlisting}

\begin{center}
\begin{tabular}{ |c|c| } 
 \hline
 reg & unsigned var of any bit size  \\ 
 reg signed & signed  \\ 
 integer & signed 32-bit var  \\ 
 real, time, realtime & Only for sims \\
 \hline
\end{tabular}
\end{center}
Can be assigned only within a procedure, a task, or a function
\begin{lstlisting}
    //reg [MSB:LSB] <signal name>;
    //reg [LSB:MSB] <signal name>;
    reg[7:0] out;
    integer count;
\end{lstlisting}

\item Instantiation Format
\begin{lstlisting}
    <component_name> #<delay> <instance_name> (port_list);
\end{lstlisting}
\begin{center}
\begin{tabular}{ |c|c| } 
 \hline
 component name & Module Name\\
 delay & Optional delay through component\\
 instance name & Unique name applied to instance\\
 port list & List of signals to connect\\
 \hline
\end{tabular}
\end{center}



\end{itemize}

\end{document}
